\documentclass[A4]{scrartcl}

\usepackage{amsmath}
\usepackage{xcolor}

% Title Page
\title{}
\author{}


\begin{document}
%\maketitle
%
%\begin{abstract}
%\end{abstract}

Poisson problem
\begin{align}
	- \Delta u = f \text{ in } \Omega \\
	u = \bar{u} \text{ on } \Gamma
\end{align}

Variational form by multiplication by the test function $v$, integration over the domain $\Omega$ and perform integration by parts

\begin{equation}
	\left(\nabla v,\nabla  u\right)_\Omega - \left(v, \nabla u \cdot \boldsymbol{n} \right)_{\Gamma} = \left(v, f\right)_\Omega
\end{equation}

% REFERENCE: https://jsdokken.com/dolfinx-tutorial/chapter1/nitsche.html

Additional terms through Nitsche's enforcement of Dirichlet boundary conditions:

\begin{equation}
	- \left(\nabla v\cdot \boldsymbol{n}, u - \bar{u} \right)_{\Gamma} + \frac{\alpha}{h} \left(v, u - \bar{u} \right)_{\Gamma}
\end{equation}

This leads to the bilinear form:

\begin{equation}
	\underbrace{
	\left(\nabla v,\nabla  u\right)_\Omega
	\textcolor{blue}{ - \left(v, \nabla u \cdot \boldsymbol{n} \right)_{\Gamma} -\left(\nabla v\cdot \boldsymbol{n}, u \right)_{\Gamma} + \frac{\alpha}{h} \left(v, u \right)_{\Gamma}}}_{A_h(u,v)} = \underbrace{\left(v, f\right)_\Omega 
	\textcolor{blue}{-\left(\nabla v\cdot \boldsymbol{n}, \bar{u} \right)_{\Gamma} + \frac{\alpha}{h} \left(v, \bar{u} \right)_{\Gamma}}}_{L_h(u,v)}
\end{equation}

% REFERENCE: deal.II step-85
To ensure coercivity of the system also for small cut cells, a so-called ghost penalty term $g_h$ can be added to the weak formulation, which reads for continuous elements as
\begin{equation}
	g_h(u,v) = \gamma_A \sum_{F\in F_h} (h_F [\nabla v \cdot \boldsymbol{n}], [\nabla u \cdot \boldsymbol{n}])
\end{equation}
with the ghost penalty parameter $\gamma_A$, $h_F$ is some measure of the face size and the jump operator $[\bullet]$.

\end{document}          
